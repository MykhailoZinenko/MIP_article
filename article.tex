% !TEX TS-program = pdflatex
% !TEX encoding = UTF-8 Unicode

% This is a simple template for a LaTeX document using the "article" class.
% See "book", "report", "letter" for other types of document.

\documentclass[11pt]{article} % use larger type; default would be 10pt

\usepackage[utf8]{inputenc} % set input encoding (not needed with XeLaTeX)

%%% Examples of Article customizations
% These packages are optional, depending whether you want the features they provide.
% See the LaTeX Companion or other references for full information.

%%% PAGE DIMENSIONS
\usepackage{geometry} % to change the page dimensions
\geometry{a4paper} % or letterpaper (US) or a5paper or....
% \geometry{margin=2in} % for example, change the margins to 2 inches all round
% \geometry{landscape} % set up the page for landscape
%   read geometry.pdf for detailed page layout information

\usepackage{graphicx} % support the \includegraphics command and options

% \usepackage[parfill]{parskip} % Activate to begin paragraphs with an empty line rather than an indent

%%% PACKAGES
\usepackage{booktabs} % for much better looking tables
\usepackage{array} % for better arrays (eg matrices) in maths
\usepackage{paralist} % very flexible & customisable lists (eg. enumerate/itemize, etc.)
\usepackage{verbatim} % adds environment for commenting out blocks of text & for better verbatim
\usepackage{subfig} % make it possible to include more than one captioned figure/table in a single float
% These packages are all incorporated in the memoir class to one degree or another...

\usepackage{cite}

%%% HEADERS & FOOTERS
\usepackage{fancyhdr} % This should be set AFTER setting up the page geometry
\pagestyle{fancy} % options: empty , plain , fancy
\renewcommand{\headrulewidth}{0pt} % customise the layout...
\lhead{}\chead{}\rhead{}
\lfoot{}\cfoot{\thepage}\rfoot{}

%%% SECTION TITLE APPEARANCE
\usepackage{sectsty}
\allsectionsfont{\sffamily\mdseries\upshape} % (See the fntguide.pdf for font help)
% (This matches ConTeXt defaults)

%%% ToC (table of contents) APPEARANCE
\usepackage[nottoc,notlof,notlot]{tocbibind} % Put the bibliography in the ToC
\usepackage[titles,subfigure]{tocloft} % Alter the style of the Table of Contents
\renewcommand{\cftsecfont}{\rmfamily\mdseries\upshape}
\renewcommand{\cftsecpagefont}{\rmfamily\mdseries\upshape} % No bold!

%%% END Article customizations

%%% The "real" document content comes below...

\title{Context-Aware Recommendation Systems}
\author{Mykhailo Zinenko}
\date{01.10.2024} % Activate to display a given date or no date (if empty),
         % otherwise the current date is printed 

\begin{document}
\maketitle

\begin{abstract}
Recommendation systems have come a long way since their early days of simple content matching or collaborative filtering to model user preferences and make recommendations. While these traiditional approaches served humanity well, they often missed the mark when it came to the subtle, real-world factors that shape our choices. Context-aware recommendation systems (CARS) have came as an advanced solution to the limitations of the traditional recommendation systems, improving personalization by using contextual information such as the users's time, location or mood. They become incredibely popular in many domains such as e-commerce, entertainment and tourism. The paper discusses how contextual factors like location, time, social surroundings, and user intent can improve the relevance and precision of recommendations. Explores the underlying mechanisms of CARS, including how they collect and analyze contextual data to enchance user experience. In addition, it aims to evaluate the benefits of context-aware systems, such as increased user satisfaction and dynamic adaptation, versus their disadvantages, for example privacy concerns and data processing complexity. The research suggests potential directions for future CARS development and how context-aware interaction can change user communication across domains.
\end{abstract}

\section{Introduction}

\section{Overview of Recommendation Systems}

\section{Understanding Context-Aware Recommendation Systems (CARS)}

\section{Contextual Factors and Their Impact}

\section{Mechanisms of Context-Aware Systems}

\section{CARS: Benefits and Challenges}

\section{Conclusion}

\bibliography{literature}
\bibliographystyle{plain}
\end{document}
